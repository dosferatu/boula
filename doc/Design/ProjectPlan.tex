\documentclass[]{report}   % list options between brackets

% list packages between braces
\usepackage{titlesec}             % http://ctan.org/pkg/titlesec 

% For changing the table of content entries in to hyperlinks
%\usepackage[linktoc=all]{hyperref}
%\hypersetup{colorlinks=true, linkcolor=blue}

% type user-defined commands here
\renewcommand{\thesection}{}% Remove section references...
\renewcommand{\thesubsection}{\arabic{subsection}}%... from subsections

\begin{document}
\raggedright{}  % Don't allow text to be spread to the right margin

\title{PCIe to OCP Bridge \\
  Project Development Plan
}   % type title between braces

\author{Kevin Bedrossian \\
  Peter Depeche \\        % type author(s) between braces
  Benjamin Hunstman \\
  Michael Walton
}

\date{January 10, 2014}    % type date between braces
\maketitle

\begin{abstract}
  PCI Express (PCIe) was created in 2004 and became into wide spread use soon after that due to the great benefits over PCI\@.
  It allows for increased bandwidth and flexibility that could not be achieved before\@.
  Revisions of the PCIe protocol continue to be developed giving increased bandwidth and functionality to the already robust communications bus\@.
  Open Core Protocol (OCP) is an openly licensed, core-centric protocol intended to meet system level integration challenges\@.
  It is an independent bus interface for on-chip systems for communications\@.
  These two protocols will be used to develop a System On Chip (SOC) that will allow for direct communication of the PCIe bus with a Virtual File System (VFS) supporting all memory transactions\@.
  This is to be accomplished by an IP block that will transparently bridge these two standards. The target device will be a Xilinx Spartan-6.
\end{abstract}

\tableofcontents

\chapter{Introduction}
A PCIe to OCP bridge IP block will be implemented to allow communication between a VFS\@.
The design will be validated by performing Direct Memory Access (DMA) communication with an EMMC memory device connected to the FPGA platform.

\chapter{Design Plan}
The following is the project plan that will be used in developing the bridge using a Hardware Design Language (HDL) such as Verilog and System Verilog.
\linebreak

\subsection{Design}
\begin{itemize}
  \item{Pseudo-code of core and test bench}
  \item{Write HDL/Verilog for simulation that is technology independent}
  \item{Analyze for functionality}
  \item{Revise HDL/Verilog design}
\end{itemize}

\subsection{Optimization}
\begin{itemize}
  \item{Map HDL to Xilinx Vertex 6 logic circuits and functional blocks}
  \item{Optimize for technology dependence}
    \begin{itemize}
      \item{Prepare for synthesis}
    \end{itemize}
  \item{Timing Analysis}
    \begin{itemize}
      \item{Check for speed, setup, and hold time}
      \item{Form constraints}
    \end{itemize}
  \item{Revise HDL}
\end{itemize}

\subsection{Place \& Route}
\begin{itemize}
  \item{Map (place) netlist to FPGA}
  \item{Route the structures (cores) on the FPGA to interconnect and perform the desired function}
  \item{Timing Analysis}
    \begin{itemize}
      \item{Apply further constraints if needed}
      \item{Move blocks around on the FPGA using a floor planning tool if needed to help timing}
    \end{itemize}
  \item{Revise HDL}
\end{itemize}

\subsection{Validation}
\begin{itemize}
  \item{Check design functionality}
  \item{Check if design meets performance goals}
\end{itemize}

\chapter{Test Plan}
This is a rough draft of the actual test bench that will be used to verify the bridge at each step of the design process.
\linebreak

\subsection{Design}
\begin{itemize}
  \item{TBD --- Unit tests for design will be included in the design process}
\end{itemize}

\subsection{Optimization}
\begin{itemize}
    \item{TBD --- Validation for timing requirements and functionality after changes mandated by synthesis requirements have been implemented}
\end{itemize}
\subsection{Place \& Route}
\begin{itemize}
  \item{TBD --- Validation for timing requirements using new analysis results and test for functionality after implementing hardware specific changes}
\end{itemize}
\subsection{Verification}
\begin{itemize}
  \item{Test 1}
    \begin{itemize}
      \item{Successful read of a single word from flash}
      \item{Successful write of a single word to flash}
    \end{itemize}
  \item{Test 2}
    \begin{itemize}
      \item{Successful read of a burst from flash}
      \item{Successful write of a burst to flash}
    \end{itemize}
  \item{Test 3}
    \begin{itemize}
      \item{Successful completion of ongoing DMA communications with flash}
    \end{itemize}
\end{itemize}

\chapter{Questions}
The following is a list of questions that need to be answered to help facilitate the project in moving forward:
\linebreak

\begin{enumerate}
  \item{Can we get the FPGA board now?}
    \begin{enumerate}
      \item{Needed to test out the PCIe Core Gen from Xilinx to ensure a complete understanding of what we are working with}
    \end{enumerate}
  \item{What exactly is provided for us in the system and has it been tested?}
    \begin{enumerate}
      \item{PCIe Core Gen}
      \item{EMMC}
    \end{enumerate}
  \item{What version of PCIe are we to target (1.0, 2.0, etc)?}
  \item{Is there someone that can guide us or provide us with some direction or suggestions?}
\end{enumerate}

%\begin{thebibliography}{9}
  % type bibliography here
%\end{thebibliography}

\end{document}
